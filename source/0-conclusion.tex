\section{Заключение}
В рамках данной работы были рассмотрены и применены методы машинного обучения с учителем к задаче определения неизвестных демографических атрибутов пользователя. Эксперименты были проведены для таких характеристик как возраст и пол. 

Также были рассмотрены и использованы алгоритмы для агрегирования информации из структуры социального графа, которая представляется в виде конечномерных векторов.  

В данной работе было показано, что векторные представления вершин социального графа могут являться довольно значимыми признаками для определения демографических атрибутов и следовательно есть основания применять их для данной задачи, в том числе совместно с
признаками, извлекаемыми из профилей пользователей.  Однако проблемой использования данных признаков является то, что их вычисление, как правило, вычислительно затратно, поэтому при изменении структуры социального графа переобучение модели занимает продолжительное время. 

С вышеупомянутой проблемой прекрасно справляется графовая нейронная сеть, которая не стремится обучать векторное представления для каждого узла, а напротив обучает набор агрегирующих функций, которые способны преобразовать исходные данные узла в векторное представление.

В качестве продолжения данной работы можно рассмотреть применение текстовой информации из постов пользователей в дополнении к информации, получаемой из структуры социального графа. Также можно
сформулировать и решить задачу одновременного определения нескольких атрибутов пользователей, поскольку между разными демографическими атрибутами часто существуют неявные связи, которые могут улучшить качество определения скрытого атрибута.


\clearpage