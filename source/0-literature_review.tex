\section{Обзор литературы}

Большинство работ, на сегодняшний день, используют данные из Twitter и Facebook и основываются на содержимом сообщений пользователей. Например, в работах \cite{Learning multiview embeddings of twitter users, Developing age and gender predictive lexica over social media, Cross-platform differences in self-disclosure and trait prediction} используется техника базирующая на мешке слов (bag of words - BOW). В этой технике предполагается, что текст пользователя представляется как неупорядоченный набор слов и в качестве признаков используются n-граммы и для каждой такой n-граммы подсчитывается абсолютная или относительная часта. 

В работе \cite{Inferring perceived demographics from user emotional tone and user-environment emotional contrast} также представлен интересный подход, который основывается на эмоциональном тоне пользователя и эмоциональном контрасте пользовательской среды. Эксперименты показали, что добавления этих признаков к обычным признакам, извлеченным из текстов сообщений, дает неплохое улучшение качества модели.

Однако, проблема подходов, которые базируются только на текстовой информации пользователя в том, что они не учитывают взаимоотношений между пользователями. Поэтому следуя идее обучения представлению и успеху  в применении векторных представлений слов \cite{Efficient estimation of word representations in vector space}, был предложен метод DeepWalk \cite{DeepWalk}, который рассматривается как первый метод графовых векторных представлений. Похожие подходы, такие как node2vec \cite{Node2Vec}, LINE  \cite{LINE} и SDNE \cite{SDNE}, также достигли прорывов. Однако эти методы могут быть вычислительно дорогими и неоптимальными для больших графов. Примером, использования вышеупомянутых подходов, является метод, описанный в статье \cite{Exact age prediction in social networks}, который использует векторные представления пользователей с помощью алгоритма DeepWalk  и применяет к этим представлениям модель линейной регрессии. 

Для решения проблем, которым подвержены графовые векторные представления стали разрабатываться графовые нейронные сети (graph neural network - GNN) \cite{Graph neural networks: A review of methods and applications}. Они основываются на сверточных нейронных сетях (convolutional neural network - CNN) и векторных представлениях графов и предназначены для коллективного агрегирования информации из структуры графа. Примером алгоритма, который внес свой вклад в данную работу может послужить метод GraphSAGE, который описан в статье \cite{GraphSAGE}.






\clearpage