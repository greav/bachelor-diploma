\section{Глава 3. Графовые эмбеддинги}
В данной части будут рассмотрены алгоритмы векторизации вершин социального графа, а также их преимущества и недостатки.

\subsection{3.1. DeepWalk}

DeepWalk \cite{DeepWalk} -- это один из самых первых алгоритмов, который создает векторное представление узлов в графе с помощью случайных блужданий, а именно - идея заключается в переходе от структуры графа к некоторой последовательности вершин, которая сохранит основные свойства и информацию о графе. 

Данный метод является простым обобщением концепции word2vec \cite{word2vec} на графы, поскольку вершины в данном алгоритме можно рассматривать как слова, а случайные блуждания - как предложения. В итоге при обучении к сгенерированным случайным блужданиям применяется аналогичная техника, как и в word2vec, а именно -- SkipGram модель.  

Векторные представления вершин, найденные этим алгоритмом должны удовлетворять следующим 4 свойствам:

\begin{enumerate}
\item Адаптируемость. Т.е. векторные представления вершин социального графа не должны вынуждать нас пересчитывать их при добавлении новых социальных связей.
\item Социальная осведомленность. Расстояние между найденными скрытыми представлениями вершин социальной сети должно служить метрикой схожести пользователей.
\item Малое число измерений скрытых представлений. При мало количестве размещенных данных, модели с меньшим число измерений гораздо лучше обобщают данные и ускоряют сходимость алгоритма.
\item Непрерывность. \textbf{TODO}
\end{enumerate}

Решение, предложенное авторами оригинальной статьи удовлетворяет всем вышеприведенным характеристикам.






\clearpage