\section{Введение}
Стремительный рост социальных сетей привел к огромным ежедневно генерируемым пользователями потокам данных. И, как оказалось, информация, извлекаемая из большого количества свободнодоступного публичного контента потенциально может выявить многие черты, предпочтения и мнения владельца профиля.

Подъем сервисов социальных сетей привел к растущему потенциалу
для персонализации в компьютерных системах, начиная от
интеллектуальных пользовательских интерфейсов или диалоговых агентов и
систем рекомендаций до крупномасштабной аналитики здравоохранения,
опроса в режиме реального времени, онлайн-рекламы и маркетинга.
Исследователи начали добывать массивные объемы персонализированных и
разнообразных данных, полученных в социальных сетях, с целью изучения
демографических характеристик пользователей, таких как пол, возраст,
политические предпочтения, пользовательские
интересы, а также эмоции, психо-дмографический профиль и мнения, которые они выражают. В результате
было реализовано несколько интеллектуальных аналитических услуг в
социальных сетях \cite{ApplyMagicSauce, PersonalityInsights}. Эти службы принимают на вход профиль из социальной
сети и выводят прогнозы о личности, эмоциях, настроениях и
демографических и характеристиках человека, владеющего профилем.

Вывод демографических характеристик из социальных сетей является полезным механизмом, позволяющим лучше понять свою аудиторию и облегчить взаимодействие с этой аудиторией.  На сегодняшний день, общим подходом к определению демографических характеристик является использование методов машинного обучения с учителем, обученных по текстовым признакам. Однако, основным ограничением этого подхода является то, что он мало использует топологию сети. Поэтому для борьбы с ограничениями этого подхода предложены методы, базирующиеся на векторном представлении вершин графов и подходы, которые используют нейронные сети для изучения общей структуры социального графа.

\clearpage