\section{Глава 1. Сбор и подготовка данных}

В данной главе будет рассмотрен способ сбора данных, краткая характеристика этих данных, а также предобработка полученных данных.

\subsection{1.1. Сбор данных}

Для сбора информации о пользователях в качестве соцаильной сети была выбрана социальная сеть ВКонтакте. Данная сеть является наиболее популярной в русскоязычном сегменте и ее главным преимуществом является удобный интерфейс прикладного программирования (application programming interface - API) для разработчиков. API - это интерфейс, предоставляющий набор готовых функций, процедур и классов, для удобного взаимодействия разработчиков с сервисом. 

В качестве атрибутов пользователей были выбраны следующие признаки, извлекаемые из профилей:
\begin{enumerate}
\item Пол
\item Возраст
\item Город проживания
\item Страна проживания
\item Текущая деятельность
\item Политическая принадлежность
\item Количество детей
\item Семейное положение
\item Город расположения школы
\item Год окончания школы
\item Город расположения университета
\item Год окончания университета
\item Уровень образования
\end{enumerate}

Стоит сделать замечание по поводу используемых признаков. Так, например, под городом проживания, страной проживания, городом расположения школы и университета здесь понимаются целочисленные идентификаторы. Возраст, количество детей, год окончания школы и университета принимают целочисленное значение. Все остальные атрибуты являются категориальными, а именно:
\begin{itemize}
\item Пол $\in \{\text{мужской, женский}\}$
\item Текущая деятельность $\in$ \{работа, среднее образование, высшее образование\}
\item Политические предпочтения $\in$ \{коммунистические, социалистические, умеренные, либеральные, консервативные, монархические, ультраконсервативные, индифферентные, либертарианские\}
\item Семейное положение $\in$ \{женат/замужем, не женат/не замужем, неизвестно\}
\item Уровень образования $\in$ \{среднее образование, высшее образование, неизвестно\}
\end{itemize}


Для того, чтобы загрузить информацию о профилях пользователей, а также информацию о социальном графе,  был реализован программный модуль, посредством языка python3.7, который использует API ВКонтакте \cite{API VK}. Главным ограничением API ВКонтакте является то, что он позволяет делать не более 3 запросов в секунду.

Реализованный программный модуль делает следующие задачи:
\begin{enumerate}
\item Случайным образом генерируются пользовательские идентификаторы 
\item Для полученного списка пользователей проверяется, указал ли он пол и возраст с помощью метода API <<users.get>>
\item Собирается информация о списке друзей, путем обращения к методу API <<friends.get>> и затем исходная выборка фильтруется на предмет того, чтобы у каждого пользователя было не менее 25 друзей
\item Для итогового списка пользователей собирается информация о профилях с помощью метода API <<users.get>>
\end{enumerate}

В итоге были получены социальные связи для 320 тысяч пользователей и демография для 23 миллионов.

Все данные было принято хранить в формате CSV, поскольку в этом формате данные хранить удобнее для библиотек машинного обучения и эффективнее по памяти, чем в JSON.
В итоге данные о всей демографии пользователей и социальных связей заняли примерно 2 ГБ.

\subsection{1.2. Предобработка данных}



\clearpage