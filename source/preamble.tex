\usepackage[utf8]{inputenc}
\usepackage[russian]{babel}
\usepackage{amsmath,amsfonts,amssymb,cite,url,verbatim,graphicx,wrapfig}
\graphicspath{{images/}}
\usepackage{booktabs}
\usepackage{graphicx}
\usepackage{array}
\usepackage{relsize}
\usepackage[ruled, linesnumbered]{algorithm2e}

\SetKwInput{KwData}{Вход}
\SetKwInput{KwResult}{Выход}
\SetKwInput{KwIn}{Входные данные}
\SetKwInput{KwOut}{Выходные данные}
\SetKwIF{If}{ElseIf}{Else}{если}{тогда}{иначе если}{иначе}{конец условия}
\SetKwFor{While}{до тех пор, пока}{выполнять}{конец цикла}
\SetKw{KwTo}{от}
\SetKw{KwRet}{возвратить}
\SetKw{Return}{возвратить}
\SetKwBlock{Begin}{начало блока}{конец блока}
\SetKwSwitch{Switch}{Case}{Other}{Проверить значение}{и выполнить}{вариант}{в противном случае}{конец варианта}{конец проверки значений}
\SetKwFor{For}{цикл}{выполнять}{конец цикла}
\SetKwFor{ForEach}{для всех}{выполнять}{}
\SetKwRepeat{Repeat}{повторять}{до тех пор, пока}
\SetAlgorithmName{Алгоритм}{алгоритм}{Список алгоритмов}


% Поля: верхнее – 2 см, нижнее – 2 см, левое – 3 см, правое – 1.5 см.
\usepackage{geometry}
\geometry{
	left = 3cm,
	top = 2cm,
	right = 1.5cm,
	bottom = 2cm
}


%\setlength{\parindent}{1.5cm}

% Кегль: основной текст – 14 пт, названия параграфов – 16 пт, названия глав – 18 пт, текст в таблице, подписи к рисункам, таблицам – 12 пт.
\renewcommand{\small}{\fontsize{12}{12}\selectfont}
\renewcommand{\normalsize}{\fontsize{14}{14}\selectfont}
\renewcommand{\large}{\fontsize{16}{16}\selectfont}
\renewcommand{\Large}{\fontsize{18}{18}\selectfont}
\renewcommand{\huge}{\fontsize{20}{20}\selectfont}
\usepackage{sectsty}
\sectionfont{\Large}
\subsectionfont{\large}
\paragraphfont{\normalsize}
\usepackage[font=small]{caption}
\usepackage{subcaption}

% Межстрочный интервал: 1.5 строки.
\usepackage{setspace}
\linespread{1.5}

% Абзацный отступ. Первая строка каждого абзаца должна иметь абзацный отступ 1.25 см.
\usepackage{indentfirst}
\setlength{\parindent}{1.25cm}

% Выравнивание основного текста по ширине поля.
\usepackage{ragged2e}
\justifying

\renewcommand\thesubfigure{\asbuk{subfigure}} %подписываем подграфики кириллицей

%%% здесь лучше ничего не трогать
\setcounter{tocdepth}{2}
\renewcommand{\thesection}{}
\renewcommand{\thesubsection}{}
\allsectionsfont{\centering}
\usepackage{titlesec}
\newcommand{\sectionbreak}{\clearpage}
\usepackage{tocloft}
\cftsetindents{section}{0em}{0em}
\cftsetindents{subsection}{2em}{0em}
\AtBeginDocument{\renewcommand{\contentsname}{\begin{center} \vskip-2.5cm \Large{Содержание} \end{center}}}
\AtBeginDocument{\renewcommand{\bibname}{}}
%%%